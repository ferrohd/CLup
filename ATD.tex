% Options for packages loaded elsewhere
\PassOptionsToPackage{unicode}{hyperref}
\PassOptionsToPackage{hyphens}{url}
%
\documentclass[10pt, a4page]{report}
\usepackage{amsmath,amssymb}
\usepackage{lmodern}
\usepackage{ifxetex,ifluatex}
\usepackage{float}
\usepackage{caption}
\usepackage{subcaption}
\ifnum 0\ifxetex 1\fi\ifluatex 1\fi=0 % if pdftex
  \usepackage[T1]{fontenc}
  \usepackage[utf8]{inputenc}
  \usepackage{textcomp} % provide euro and other symbols
\else % if luatex or xetex
  \usepackage{unicode-math}
  \defaultfontfeatures{Scale=MatchLowercase}
  \defaultfontfeatures[\rmfamily]{Ligatures=TeX,Scale=1}
\fi
% Use upquote if available, for straight quotes in verbatim environments
\IfFileExists{upquote.sty}{\usepackage{upquote}}{}
\IfFileExists{microtype.sty}{% use microtype if available
  \usepackage[]{microtype}
  \UseMicrotypeSet[protrusion]{basicmath} % disable protrusion for tt fonts
}{}
\makeatletter
\@ifundefined{KOMAClassName}{% if non-KOMA class
  \IfFileExists{parskip.sty}{%
    \usepackage{parskip}
  }{% else
    \setlength{\parindent}{0pt}
    \setlength{\parskip}{6pt plus 2pt minus 1pt}}
}{% if KOMA class
  \KOMAoptions{parskip=half}}
\makeatother
\usepackage{xcolor}
\IfFileExists{xurl.sty}{\usepackage{xurl}}{} % add URL line breaks if available
\IfFileExists{bookmark.sty}{\usepackage{bookmark}}{\usepackage{hyperref}}
\hypersetup{
  hidelinks,
  pdfcreator={LaTeX via pandoc}}
\urlstyle{same} % disable monospaced font for URLs
\usepackage{longtable,booktabs,array}
\usepackage{calc} % for calculating minipage widths
% Correct order of tables after \paragraph or \subparagraph
\usepackage{etoolbox}
\makeatletter
\patchcmd\longtable{\par}{\if@noskipsec\mbox{}\fi\par}{}{}
\makeatother
% Allow footnotes in longtable head/foot
\IfFileExists{footnotehyper.sty}{\usepackage{footnotehyper}}{\usepackage{footnote}}
\makesavenoteenv{longtable}
\usepackage{graphicx}
\makeatletter
\def\maxwidth{\ifdim\Gin@nat@width>\linewidth\linewidth\else\Gin@nat@width\fi}
\def\maxheight{\ifdim\Gin@nat@height>\textheight\textheight\else\Gin@nat@height\fi}
\makeatother
% Scale images if necessary, so that they will not overflow the page
% margins by default, and it is still possible to overwrite the defaults
% using explicit options in \includegraphics[width, height, ...]{}
\setkeys{Gin}{width=\maxwidth,height=\maxheight,keepaspectratio}
% Set default figure placement to htbp
\makeatletter
\def\fps@figure{htbp}
\makeatother
\setlength{\emergencystretch}{3em} % prevent overfull lines
\providecommand{\tightlist}{%
  \setlength{\itemsep}{0pt}\setlength{\parskip}{0pt}}
\setcounter{secnumdepth}{-\maxdimen} % remove section numbering
\ifluatex
  \usepackage{selnolig}  % disable illegal ligatures
\fi

%Custom titling

\usepackage{titling}

% set up \maketitle to accept a new item
\pretitle{\begin{center}\placetitlepicture\Huge}
\posttitle{\par\lineskip 1em\placesubtitle\end{center}\vskip 3em}
\preauthor{\begin{center}
        \large \lineskip 3em%
        \begin{tabular}[t]{c}}
\postauthor{\end{tabular}\par\placeprofessor\end{center}}
\predate{\begin{center}\large\vskip 3em}
\postdate{\par\placeversion\par\end{center}}

% commands for including the picture
\newcommand{\titlepicture}[2][]{%
  \renewcommand\placetitlepicture{%
    \includegraphics[#1]{#2}\par\medskip
  }%
}
\newcommand{\placetitlepicture}{} % initialization

% commands for including the subtitle
\newcommand{\subtitle}[2][]{%
  \renewcommand\placesubtitle{%
    \Large #2\par\medskip
  }%
}
\newcommand{\placesubtitle}{} % initialization

% commands for including the professor
\newcommand{\professor}[2][]{%
  \renewcommand\placeprofessor{%
    \large Professor: #2\par\medskip
  }%
}
\newcommand{\placeprofessor}{} % initialization

% commands for including the version
\newcommand{\version}[2][]{%
  \renewcommand\placeversion{%
    \large Version: #2\par\medskip
  }%
}
\newcommand{\placeversion}{} % initialization
\graphicspath{ {assets/} }
\usepackage{dirtree}

\titlepicture[width=0.75\textwidth]{./polimi_logo}
\title{Acceptance Test Deliverable}
\subtitle{Customers Line-up}
\author{\href{https://github.com/ferrohd}{Alessandro Ferrara} -
\href{https://github.com/lorenzofratus}{Lorenzo Fratus}}
\professor{Elisabetta di Nitto}
\date{February 13, 2021}
\version{1.0}

\begin{document}

\maketitle

\tableofcontents

\chapter{1. Analyzed project}

\section{A. Authors}

The authors of the analyzed prototype are:
\begin{itemize}
    \item \href{https://github.com/rodrigocedeno}{Jesus Rodrigo Cedeño Jimenez}
    \item \href{https://github.com/dacumming}{Diego Andres Cumming Cortes}
    \item \href{https://github.com/anpugliese}{Angelly de Jesus Pugliese Viloria}
\end{itemize}

\section{B. Repository}

The repository of this project can be found at this \underline{\href{https://github.com/anpugliese/CedenoCummingPugliese}{link}}.

\section{C. Reference documents}

The following documents were considered during the acceptance test:
\begin{itemize}
\item
  Assignment document A.Y. 2020/2021 (``Requirement Engineering and Design Project: goal, schedule, and rules'')
\item
  Assignment document A.Y. 2020/2021 (``I\&T assignment goal, schedule, and rules'')
\item
  Requirement Analysis and Specification Document - CLup: Customers Line-up Software, version 2 (referenced as ``RASD'')
\item
  Design Document - CLup: Customers Line-up Software, version 2 (referenced as ``DD'')
\item
  Implementation and Test Deliverable - CLup: Customers Line-up Software, version 1 (referenced as ``ITD'')
\end{itemize}

\chapter{2. Installation setup}

\section{A. Installation guide}

The installation guide is split into two \emph{README} files (one for each main folder).

The guide for the installation of the back-end (\emph{server} folder) is not so clear, you need to install many dependencies and manually set environment variables.\\
From a non-poweruser perspective this is quite a lot to handle.

To use a local instance of the database, it is assumed that the user has already installed and to is able to use \emph{PostgreSQL}.\\
Unfortunately, no structure of the database is presented in the guide nor in the ITD so it is not possible to create a functioning database without looking into the DD or the code.

\section{B. Position spoofing}

As stated in the ITD (\emph{2.2. Map}), the map presents the user with the list of stores only in a radius of 2.2km.\\
Furthermore, the system does not provide for the insertion of new points of sale.

In order to use the application it is therefore necessary to ``spoof'' the position through an external plugin.

This information, essential for the functioning of the system, should have been included in the installation guide.

\chapter{3. Acceptance test cases}

Before manually testing the main functions of the application, automatic tests included in the software has been performed to verify the overall correctness of the system.
No errors has been found at this stage.

\section{A. Login and registration}

\subsection{A.1. Registration}

The registration process is straight forward, requires a username, password and acceptance of the \emph{Terms of Service and Privacy Policy}. The latter checkbox is purely aesthetic.
After clicking on the ``Continue'' button, the user is redirected to the login page. 

\subsection{A.2. Login}

From the login page is possible to insert a valid combination of username and password in order to authenticate.
After clicking on the ``Login'' button, the user is redirected to the main page of the application that displays the map.

\subsection{A.3. Input validation}

Some tests have been carried out using invalid inputs and these are the results:
\begin{itemize}
    \item Registration with correct credentials: \textbf{\emph{201 Created}, redirect to ``login''}
    \item Registration without username, password or the acceptance of the \emph{Term of Service and Privacy Policy}: \textbf{\emph{201 Created}, redirect to ``login''}
    \item Registration with an already existing username: \textbf{\emph{500 Internal Server Error}}
    \item Login with correct credentials: \textbf{\emph{200 OK}, redirect to ``supermarkets\_list''}
    \item Login with wrong credentials: \textbf{\emph{401 Unauthorized}}
    \item Login without username or password: \textbf{\emph{401 Unauthorized}}
\end{itemize}

There is neither client-side nor (apparently) server-side validation of input data, so requests with missing or incorrect data can be made.

The only operations not allowed are the registration with an already existing username and the login with incorrect (or missing) credentials, probably because those queries are rejected from the DBMS (as can be guessed from the status code of the responses).

Registration without username was possible only once as the empty string is considered a valid username and therefore a second registration would imply more users with the same username.

Unfortunately, there is no feedback when an operation is rejected by the server. This is not critical but it allows the user to better understand what is happening.

\section{B. Main page}

The main page features a map with a list of stores and, on the right, four different control buttons (one at the top and three at the bottom).

\subsection{B.1. Stores}

Stores are color coded based on waiting time. Clicking on an icon displays a popup with two buttons: ``Line up'' and ``Book''.

A strange delay in updating the store waiting time can be seen when performing operations such as queuing or canceling a ticket.

\subsubsection{B.1.1. Line up}

Clicking on the ``Line up'' button opens the QR code page showing the newly created ticket and the expected waiting time or a 5-minute countdown before the ticket expires (if the waiting time is 0).

This feature works as per the documentation.
As expected, it is not possible to line up if there is already a ticket for any shop.

\subsubsection{B.1.2. Book}

Clicking on the ``Book'' button opens a page to insert the wanted date and hour. The system allows to get a ticket from one hour to one week in advance, it would have been a good idea to point it out in the interface.

Upon confirmation, the system shows the page containing the generated QR code where a countdown called waiting time is shown.
In this case it would probably have been more appropriate to display the date and time of validity of the ticket.

As before, the function does what is expected and it is not possible to book if a ticket for any store already exists.

\subsection{B.2. Filter}

The first button in the lower right corner is the filter. Clicking on it displays a popup.

From this popup is possible to select one or more store \emph{brands} and a maximum waiting time. By clicking on ``Apply'' the filter is set and, by clicking on ``Clear'' the filter is removed, in both cases the map is updated.

This is a simple and useful function and it works as one would expect. The only drawback is that is important to standardize the store \emph{brands} in order to avoid duplication.

\subsection{B.3. Buy \emph{ASAP}}

The second button in the lower right corner opens a page with a list of stores sorted by distance and waiting time. This page allows to line up directly from the list. There are no particular comments on this feature.

\subsection{B.4. QR code}

The last button in the lower right corner opens the page that the user is redirected to after requesting a ticket.

This page contains a QR code (representing the username) and the expected waiting time (or the countdown to the ticket expiration).
The page also shows two buttons to ``Go back to map'' and to ``Cancel request''. Both buttons take the user back to the main page but the second one deletes the current ticket.

Opening the browser \emph{Network} interface it is possible to notice that every second the page sends two requests to the server: \emph{qrcode} and \emph{remainingTime}.\\
Since this is not necessarily a real-time system, it is a waste of resources to send requests this frequently. One update every minute would probably suffice.

\subsection{B.5. Logout}

In the upper right corner of the screen there is a button to log out. This feature works as expected.

\clearpage
\section{C. Security issues}

The testing activity highlighted the fact that there are no security checks on the data that the application sends to the server.

Specifically, changing the username in the body of HTTP requests made it possible to queue for another user (even if it did not exist) and also to cancel another user's ticket.

This test was not done for the booking functionality but it is safe to assume that it would behave the same way.

One solution is to verify that the author of the request matches the user who is affected. 
This can be done by leveraging the already implemented JSON Web Token (JWT).

A software that does not implement an authentication and authorization mechanism is vulnerable to malicious requests.

\chapter{4. Additional comments}

\section{A. Software compatibility}

The application does not seem to work on the latest version of the Safari web browser (14) unlike what is specified in the RASD (\emph{3.1.3. Software Interfaces}).\\
The map appears completely gray, therefore unusable. Not a big deal, it was just worth pointing out.

\section{B. Software architecture}

\subsection{B.1. Front-end}

In the DD, it is stated on several occasions (e.g. \emph{2.1. Overview}) that the client-side application runs on the user's device. 
This appears to conflict with other parts of the same document where the front end is regarded as a web server (e.g. \emph{2.4. Deployment view}).

\subsection{B.2. Server-side rendering and APIs}

The system has been divided into two servers:
\begin{itemize}
    \item The first one (\emph{clup} folder) is responsible for providing the user interfaces by means of server-side rendering (using Vue.js).
    \item The second (\emph{server} folder) is responsible for the application logic and the communications with the database.
\end{itemize}
These two server are connected via APIs.

Implementing a server-side rendering mechanism is a waste in the presence of a set of APIs as the page rendering operation can be performed directly from the client side (common practice).

\section{C. Limits of the acceptance testing}

The system has been developed assuming that each store has a physical device capable of scanning tickets before letting customers in and out.

This machine should also be able to issue tickets as a fallback for customers who are not registered in the application.

Unfortunately, this assumption limits the range of tests that can be performed on the application as currently the user experience ends with the issuance of the ticket.

\chapter{5. Effort spent}

\section{Pair programming}

\begin{longtable}[]{@{}
  >{\raggedright\arraybackslash}p{(\columnwidth - 2\tabcolsep) * \real{0.8}}
  >{\raggedleft\arraybackslash}p{(\columnwidth - 2\tabcolsep) * \real{0.2}}@{}}
\toprule
Topic & Hours \\ \addlinespace
\midrule
\endhead
General structure & 0.5h \\ \addlinespace
Acceptance testing & 3.5h \\ \addlinespace
\bottomrule
\end{longtable}

\section{Ferrara Alessandro}

\begin{longtable}[]{@{}
  >{\raggedright\arraybackslash}p{(\columnwidth - 2\tabcolsep) * \real{0.8}}
  >{\raggedleft\arraybackslash}p{(\columnwidth - 2\tabcolsep) * \real{0.2}}@{}}
\toprule
Topic & Hours \\ \addlinespace
\midrule
\endhead
Code analysis & 2h \\ \addlinespace
\bottomrule
\end{longtable}

\section{Fratus Lorenzo}

\begin{longtable}[]{@{}
  >{\raggedright\arraybackslash}p{(\columnwidth - 2\tabcolsep) * \real{0.8}}
  >{\raggedleft\arraybackslash}p{(\columnwidth - 2\tabcolsep) * \real{0.2}}@{}}
\toprule
Topic & Hours \\ \addlinespace
\midrule
\endhead
Documentation analysis & 2h \\ \addlinespace
\bottomrule
\end{longtable}

\end{document}